
\begin{spacing}{1.2}
	\chapter{INTRODUCTION}
\end{spacing}


\pagenumbering{arabic}
\vspace{4ex}

\section{Background}
\label{sec1:Introduction}

% elderly secara umum
The global elderly population is experiencing unprecedented growth, and this trend is expected to continue into the foreseeable future. As of 2019, there were approximately 703 million individuals aged 65 or older worldwide, accounting for 9\% of the total global population \cite{nations2019world}. Projections indicate that this number will double by the year 2050, highlighting the increasing importance of addressing the needs of this demographic. One significant challenge faced by the elderly is a decline in productivity, particularly in terms of physical capabilities, which can greatly affect their independence and quality of life.

% penurunan kemampuan fisik elderly
The decline in physical capabilities among the elderly can be mitigated through regular exercise and physical activity. However, such activities are often neglected by the elderly due to various barriers, including lack of motivation, fear of injury and limited access to professional guidance. Exercise for the elderly requires special considerations, as their physical capacities differ significantly from those of younger, more productive individuals. Consequently, specialized exercise programs designed specifically for the elderly are essential. These programs often necessitate the assistance of physiotherapists or other trained professionals to ensure that exercises are performed safely and effectively.

% isu fisik yang sering menyerang
An elderly person can experience a variety of physical issues. Some physical issues that often affect elders are frozen shoulder \cite{FrozShoulder5}, tennis elbow \cite{TennisElbow3}, and knee pain \cite{KneePain5}. Adhesive Copsufitis, often known as frozen shoulder, is a painful condition in which the shoulder joint sustains damage without causing damage to the soft tissues. This type of injury affects 2-5\% of the population on average, with women accounting for 60\% of all injuries. According to recent research, people with diabetes have a five-fold increased risk of developing frozen shoulder in their 40s to 60s when compared to the general population. The shoulder joint's wide range of motion and heavy movement loads over it make it more prone to injury. This specific joint is dependent on the Deltoid Major, a large muscle, which is supported by the Cuff Rotators, a smaller set of muscles that are also essential for joint stability and muscular activity \cite{FrozShoulder6}.

Lateral epicondylitis or better known as tennis elbow is a condition that causes pain in the elbow due to inflammation of the tendons in the upper arm. This pain tends to be constant and causes disability in the elbow, especially the radio humeral joint which is known as lateral epicondylitis or lateral epicondylalgia. The disease often occurs due to repetitive activities involving the wrist and arm. The occurrence of this disease is also in line with a person's age. The disease affects the general population at about 1-3\% and increases sharply to 19\% in subjects aged 30-60 years. This physical issue is more prone to affect women and lasts longer. The average duration of this physical issue is about 6 months to 24 months \cite{TennisElbow4}.

Knee pain is a common knee problem experienced by individuals with routine or excessive activity on the knee. It often affects athletes to the elderly. Knee pain is caused by various conditions and affects various structures around the knee, including bones, muscles, tendons and ligaments. The frequency and severity of knee pain increases when the disease affects the elderly aged 50 years and above \cite{KneePain7}. Increased severity of knee pain is associated with more serious problems. There is a greater risk of falls and hip fractures \cite{KneePain6}.
% Knee pain merupakan masalah lutut yang umum dialami oleh individu dengan aktifitas rutin atau berlebih pada bagian lutut. Penyakit ini sering menyerang seorang atlit hingga elderly. Knee pain disebabkan oleh berbagai kondisi dan mempengaruhi berbagai struktur di sekitar lutut, termasuk tulang, otot, tendon, dan ligamen. Frekuensi dan tingkat keparahan knee pain meningkat ketika penyakit ini menyerang elderly berusia 50 tahun ke atas \cite{KneePain7}. Meningkatnya keparahan knee pain dikaitkan dengan masalah yang lebih serius. Kemungkinan risiko jatuh dan patah tulang pinggul menjadi lebih besar \cite{KneePain6}.

Ariyani et al. \cite{Heuristic} developed a heuristic-based pose detection application system for recognizing elderly activities such as standing, sitting, and lying down. This research successfully leveraged human pose estimation to improve the accuracy of elderly activity recognition. Mediapipe Pose Estimation (MPE), an open-source cross-platform framework provided by Google, is one of the frameworks used for human pose estimation. MPE estimates 2D human joint coordinates in each image frame, utilizing the BlazePose architecture \cite{BlazePose}. BlazePose, a lightweight convolutional architecture, is designed for real-time pose estimation, achieving a frame rate of 10 FPS for the BlazePose Full architecture and 31 FPS for the BlazePose Lite architecture on a single mid-tier phone CPU \cite{BlazeFace}.

Convolutional Neural Networks (CNNs) have proven effective for image processing and recognition tasks, offering high accuracy in recognizing activities from images. CNNs are a type of deep learning model specifically designed to process and analyze visual data. They are composed of multiple layers, including convolutional layers, pooling layers, and fully connected layers. The convolutional layers apply filters to the input images, creating feature maps that capture various aspects of the images such as edges, textures, and shapes. Pooling layers then reduce the spatial dimensions of the feature maps, which helps to minimize the computational load and reduce the risk of overfitting. Finally, fully connected layers integrate the extracted features to make predictions. CNNs are particularly effective at recognizing patterns and objects within images, making them ideal for tasks such as image classification, object detection, and activity recognition.

Building on this, Ordóñez et al. \cite{elderly3} combined Deep Convolutional Neural Networks with Long Short-Term Memory (LSTM) networks for human activity recognition. This combination resulted in higher accuracy compared to previous methods. LSTM networks are an advanced form of Recurrent Neural Networks (RNNs), designed to overcome the limitations of traditional RNNs, which often struggle with retaining information over long sequences. LSTMs address this issue by incorporating special units called memory cells, which can store information for extended periods. These memory cells are regulated by three types of gates: the input gate, the forget gate, and the output gate. The input gate controls the addition of new information to the memory cell, the forget gate determines which information should be discarded, and the output gate manages the retrieval of information from the memory cell.

Activity recognition research for the elderly has advanced considerably in recent years. One notable study by Gochoo et al. \cite{elderly2} proposed a Deep Convolutional Neural Network (DCNN) classification approach to detect basic activities such as Bed\_to\_Toilet, Eating, Meal\_Preparation, and Relaxation. This approach demonstrated the potential of using deep learning techniques for elderly activity recognition. Similarly, Xu et al. \cite{elderly1} introduced a two-stage method for recognizing activities in elderly homes based on random forest and activity similarity, successfully identifying basic activities performed by seniors.

The integration of CNNs and LSTMs leverages the strengths of both architectures: CNNs excel at extracting spatial features from images, while LSTMs are adept at capturing temporal dependencies in sequential data. In the context of human activity recognition, CNNs can process the visual information from image sequences to identify relevant features, such as the posture and movement of individuals. These features are then fed into the LSTM network, which analyzes the temporal relationships between the sequences to recognize and classify different activities. This synergistic approach enhances the overall accuracy of the recognition system, making it more robust and reliable for real-time applications.

Adapting physical activities to the needs of the elderly is crucial to ensure they can maintain their movement routines and keep their body parts active. Simple exercises tailored to the elderly can provide a Tailoring physical activity to the needs of older adults is essential to ensure they can maintain their movement routines and keep their body parts active. Safeguarding their health by ensuring they perform exercise activities correctly is a focus of research that needs to be deepened. Simple exercises tailored to older adults can be a solution, allowing them to perform physical activities safely and effectively. Ensuring that these exercises are performed correctly is critical to preventing injury and increasing the health benefits of physical activity. In order to be able to recognize each exercise activity performed by the elderly, a deep learning model is needed that is able to properly classify each activity performed by the elderly. This is to ensure that the exercise activities performed are correct and reduce the risk of injury., allowing them to engage in physical activity safely and effectively. Ensuring that these exercises are performed correctly is essential to prevent injuries and promote the health benefits of physical activity.
% Menyesuaikan aktivitas fisik dengan kebutuhan lansia sangat penting untuk memastikan mereka dapat mempertahankan rutinitas gerak dan menjaga bagian tubuh mereka tetap aktif. Menjaga kesehatan mereka dengan memastikan melakukan aktifitas latihan dengan benar merupakan fokus riset yang perlu diperdalam. Latihan sederhana yang disesuaikan dengan lansia dapat menjadi solusi, sehingga mereka dapat melakukan aktivitas fisik dengan aman dan efektif. Memastikan bahwa latihan-latihan ini dilakukan dengan benar sangat penting untuk mencegah cedera dan meningkatkan manfaat kesehatan dari aktivitas fisik. Agar dapat mengenali setiap aktifitas latihan yang dilakukan oleh elderly, perlu sebuah model deep learning yang mampu mengklasifkasikan dengan baik setiap aktifitas yang dilakuka elderly. Hal ini untuk memastikan aktifitas latihan yang dilakukan telah benar dan mengurangi risiko cedera.


The focus of this research is on the classification of exercise activities in the elderly. The proposed exercises are tailored to address common physical issues experienced by seniors. The dataset used consists of images representing sequences of each exercise type. Exercise classification is performed using the MediaPipe Pose Estimation (MPE) framework, and the sequences of exercise activities are trained using the CNN-LSTM method. This method enables the labeling and classification of different exercise types. The resulting model will be evaluated for its accuracy in recognizing and classifying elderly exercise activities. Additionally, the research introduces PhysioExercise, a dataset that presents elderly exercise activities based on common physical issues experienced by elderly.

% In summary, addressing the physical activity needs of the growing elderly population is a pressing concern. Research in activity recognition and exercise classification, supported by advanced techniques such as CNNs, LSTMs, and human pose estimation frameworks like MPE, offers promising solutions to enhance the health and independence of the elderly. By developing specialized exercise programs and improving access to professional guidance, we can support the elderly in maintaining an active and healthy lifestyle.

\section{Formulation of the Problems}
\label{sec1:Problems}
Based on the background previously described, the problems in this study can be formulated. Currently, there are limitations on exercise activity datasets for the elderly. The exercise must be adapted to their physical problems. There are three physical issues that are often experienced by the elderly, namely frozen shoulder, tennis elbow, and knee pain. The dataset collection contains exercise activities that can prevent these physical issues. In addition, the classification of these exercises needs to be done using deep learning methods. The utilization of deep learning model in classification task is necessary considering its good capability in classification task. % Berdasarkan latar belakang yang telah diuraikan sebelumnya, maka dapat dirumuskan permasalahan dalam penelitian ini. Saat ini, terdapat keterbatasan pada dataset aktivitas olahraga untuk lansia. Latihan tersebut harus dapat disesuaikan dengan masalah fisik mereka. Terdapat tiga isu-isu fisik yang sering dialami oleh elderly, yaitu frozen shoulder, tennis elbow, dan knee pain. Pengumpulan dataset berisikan aktifitas-aktifitas latihan yang mampu mencegah munculnya isu-isu fisik tersebut. Selain itu, klasifikasi latihan-latihan tersebut perlu dilakukan dengan menggunakan metode deep learning. Pemanfaatan model deep learning dalam tugas klasifikasi perlu diadakan mempertimbangkan kemampuannya yang baik dalam tugas klasifikasi. 

\section{Objectives}
The objective of this research is to develop a The purpose of this research is to create a dataset that contains training activities in solving some physical issues that are often experienced by the elderly. These physical issues are frozen shoulder, tennis elbow, and knee pain. Each proposed exercise activity is a movement that can prevent the appearance of these physical issues. In addition, this research develops a model for classification of exercise activities for the elderly using deep learning methods based on the dataset that has been created. We implemented several methods to get outstanding results. Each model that has been generated is then analyzed for its performance to get the model with the most optimal elderly exercise activity classification ability. for the classification of exercise activities for the elderly using deep learning methods. We implement several methods to get an outstanding result. In addition, we created exercise activities for the elderly dataset, where the dataset is adaptable to elderly physical issues.
% Tujuan dari penelitan ini adalah membuat dataset yang berisikan aktifitas latihan dalam menyelesaikan beberapa isu fisik yang sering dialami oleh elderly. Isu-isu fisik tersebut adalah frozen shoulder, tenis elbow, dan knee pain. Setiap aktifitas latihan yang diusulkan adalah gerakan yang mampu mencegah munculnya isu-isu fisik tersebut. Selain itu, penelitian ini mengembangkan model untuk klasifikasi aktivitas olahraga untuk lansia menggunakan metode deep learning berdasarkan dataset yang telah dibuat. Kami mengimplementasikan beberapa metode untuk mendapatkan hasil yang luar biasa. Setiap model yang telah dihasilkan kemudian dianalisis peformanya untuk mendapatan model dengan kemampuan klasifikasi aktifitas latihan elderly paling optimal.

\section{Scope and Limitations}
In order to focus on the research goal, the limitations need to be adapted. The limitations of the research are:
\begin{itemize}
	\item The data used is a private dataset that was created in a laboratory and home environment.
	\item Since we only extract the keypoint of pose estimation, the subjects of the dataset include individuals of various ages.
	\item The exercise activities used are exercises that are adaptable to certain elderly physical issues.
	\item The exercise activities contain activity that do not require any equipment.
	\item Pose estimation is performed using MediaPipe Pose Estimation.
	\item The model is labeled and classified using the CNN, LSTM, CNN-LSTM, and deep CNN-LSTM architectures.
	\item The testing of the model exercise activities is done by some performance metrics, i.e., confusion matrix, precision, recall, f1-score, accuracy, and loss metric.
\end{itemize}

% Data yang digunakan adalah data private order of images 
% Estimasi pose dilakukan menggunakan MediaPipe Pose Estimation
% Pelabelan dan klasifikasi model menggunakan arsitektur CNN-LSTM
% Klasifikasi gerakan peregangan dibagi menjadi 3 jenis, yaitu gerakan kepala, tangan, dan kaki

\section{Contribution}
With this research, we aim to develop a classification of exercise activities for the elderly. Presenting datasets that are suitable for the physical problems of the elderly is also one of our research focuses. Specific exercises will make it easier for the elderly to maintain their health. We perform classification in several deep learning methods to show how the performance is generated for each method. We hope this work can be one of the references for improving the health of the elderly.
% Dengan riset ini, kami bertujuan untuk mengembangkan klasifikasi aktifitas latihan untuk elderly. Menyajikan dataset yang sesuai dengan permasalahan-permasalahan fisik elderly juga menjadi salah satu fokus riset kami. Gerakan yang tepat sasaran akan memudahkan elderly dalam menjaga kesehatannya. Klasifikasi kami lakukan dalam beberapa metode deep learning untuk menunjukkan bagaimana peforma yang dihasilkan untuk masing-masing metode. Kami berharap pekerjaan ini dapat menjadi salah satu rujukan untuk peningkatan kesehatan elderly.

%%%%%%%%%%%%%%%%%%%%%%%%%%%%%%%%%%%===========%%%%%%%%%%%%%%%%%%%%%%%%%%%%%%%%%%%
% Proposal Thesis 1.0
% The study of service robots has seen significant advancements in research during the past decade. Wirtz et al. \cite{SR1} defined service robots as system-based autonomous and adaptable interfaces that interact, communicate and deliver service to an organization’s customers. Service robots refer to various types of robots that provide assistance to humans in various domains. They serve individuals in many ways, including by providing them with emotional, social, and physical assistance. The design of service robots emphasizes not only the accomplishment of physical tasks but also the guiding of human-robot interaction. In certain circumstances, service robots can also offer emotional and social assistance \cite{SARs1}.

% % Pencapaian riset yang positif belakangan dekade ini telah dicapai oleh riset robot service. Wirtz et al. \cite{SR1} mendefinisikannya sebagai antarmuka yang berbasis sistem yang mandiri dan dapat beradaptasi yang berinteraksi, berkomunikasi, dan memberikan layanan kepada pelanggan organisasi. Robot service merujuk kepada berbagai macam jenis robot yang memberikan pelayanan terhadap manusia dalam berbagai domain. Robot ini memberikan berbagai tipe layanan, mulai dari layanan fisik, sosial, maupun emosional kepada manusia. Pengembangan robot service tidak hanya pada penyelesaian tugas-tugas secara fisik, namun bagaimana interaksi antara manusia dengan robot terbentuk. Di beberapa kasus, robot service juga dapat memberikan layanan emosional dan sosial \cite{SARs1}.

% % General problem
% The utilization of service robots has been focused on providing assistance to the elderly. The global elderly population is growing rapidly, and this trend is predicted to continue. In 2019, there were approximately 703 million elderly individuals aged 65 or older worldwide \cite{nations2019world} (9\% of the total global population). This number is projected to double by the year 2050. The productivity of the elderly is frequently decreased, especially in terms of physical capabilities. This condition affects their independence. One solution to address the challenges faced by the elderly is the use of service robots.

% % Pemanfaatan robot service yang menjadi salah satu fokusan adalah robot service untuk elderly. Populasi elderly di dunia tumbuh secara pesat dan tren ini diprediksi akan terus meningkat. Pada tahun 2019, ada sebanyak 730 juta elderly berusia 65 tahun atau lebih di dunia \cite{nations2019world} (9\% dari jumlah populasi manusia di dunia). Jumlah ini diprediksi akan meningkat sebanyak dua kali lipat pada tahun 2050. Elderly telah memiliki produktifitas yang menurun, terutama pada kemampuan motoriknya. Itu mempengaruhi pada kemandiriannya yang menurun pula. Salah satu penyelesaian masalah elderly ini dengan menggunakan robot service.

% % Literature Synthesis
% The development of service robots for the elderly has been the focus of several studies. Personal Robotic Aides for the Elderly (PEARL) is a project that conducts research into personal service robots for the elderly \cite{PEARL}. The objective of this project is to develop an easily utilizable personal service robot to assist older people with chronic illnesses in their everyday activities. Vargas et al. \cite{Donaxi} have proposed the Donaxi Service Robot as an assistant robot for the elderly. The Donaxi Robot Service is capable of navigating the home environment, interacting with the elderly through voice and gestures, object recognition, person recognition, and manipulation with its arm. Wang and Chen \cite{Wangrobot} have proposed a multi-purpose household auxiliary mobile robot, which is a versatile robot that facilitates the mobility of the elderly, completes physical tasks, and serves as a mobile chair. Muhtadin et al. \cite{muhtadin1} utilize the IRIS Robot, a self-developed three-wheel omnidirectional holonomic robot, as a service robot for the elderly. The IRIS Robot has automatic navigation, can map unfamiliar environments, and avoid obstacles in its surroundings. It can also assist the elderly in finding frequently misplaced items by remembering their last known location \cite{muhtadin2}. In terms of person detection, the IRIS Robot detects the elderly by estimating their pose. Human pose estimation allows for more accurate and precise recognition of elderly gestures. In service scenarios, a service robot remains in a ready state until instructions are given by the elderly. Person detection with pose estimation is used to interpret the movements and gestures of the elderly.

% One of the frameworks for human pose estimation is Mediapipe Pose Estimation (MPE). MPE is an open-source cross-platform framework provided by Google for estimating 2D human joint coordinates in each image frame \cite{MPE}. The backbone of MPE is the BlazePose architecture \cite{BlazePose}, a lightweight convolutional architecture designed for real-time pose estimation. On a single mid-tier phone CPU, the frame rate for the BlazePose Full architecture is 10 FPS while the BlazePose Lite architecture is 31 FPS \cite{BlazeFace}. However, BlazePose models have limitations in multi-person estimation. Additionally, these models estimate the pose of the first person they encounter, which can be disadvantageous for the elderly when multiple people are present. In some situations, the service robot can guess the wrong person's stand inaccurately.

% % Berbagai macam riset mengenai robot service untuk elderly telah dikembangkan. Sebuah upaya penelitian tentang robot layanan pribadi untuk lansia adalah proyek personal robotic aides for the elderly (Pearl) \cite{PEARL}. Tujuan dari proyek ini adalah menciptakan robot layanan pribadi yang dapat diakses untuk mendukung lansia dengan penyakit kronis dalam aktivitas sehari-hari mereka. Vargas \textit{et al} \cite{Donaxi} telah mengajukan Donaxi Service Robot sebagai robot asisten untuk elderly. Donaxi Robot Service mampu melakukan navigasi lingkungan dalam rumah, interaksi dengan elderly secara suara dan gestur, mengenali objek, mengenali orang, dan memiliki lengan untuk manipulasi benda. Wang and Chen \cite{Wangrobot} mengusulkan multi-purpose household auxiliary mobile robot, sebuah robot multi-bentuk yang dapat memudahkan pergerakan elderly, menyelesaikan tugas fisik, dan menjadi kursi penggerak. Muhtadin \textit{et al} \cite{muhtadin1} menggunakan IRIS Robot yang merupakan robot holonomik tiga roda omnidireksional buatan sendiri yang sedang dikembangkan sebagai layanan robot untuk lansia. IRIS Robot memiliki navigasi otomatis, mampu melakukan pemetaan di lingkungan baru yang tidak dikenal robot sebelumnya dan menghindari rintangan yang berada di lingkungannya. IRIS Robot juga dapat membantu elderly dalam menemukan barang yang sering terlupa oleh elderly letak terakhir benda-benda tersebut secara acak \cite{muhtadin2}. Pada bagian deteksi orang, IRIS Robot mendeteksi elderly dengan mengestimasi pose-nya. Human Ppse esimation mampu mengenali gestur elderly secara lebih cermat dan teliti.

% % Salah satu framework human pose estimation adalah Mediapipe Pose Estimation (MPE). MPE merupakan open-source cross-platform framework yang disediakan Google untuk mengestimasi 2D human joint coordinates di dalam setiap image frame \cite{MPE}. Menggunakan Machine Learning (ML), MediaPipe Pose membuat pipelines dan menganalisis data kognitif yang disajikan sebagai video. MPE dibangun menggunakan arsitektur BlazePose \cite{BlazePose}, sebuah jaringan arsitektur konvolusi yang ringan yang dirancang untuk mengestimasi pose secara real-time. Frame rate yang arsitektur BlazePose Full sebesar 10 FPS dan 31 FPS untuk arsitektur BlazePose Lite di dalam single mid-tier phone CPU \cite{BlazeFace}.

% % Research Problem
% % Dalam pelayanan, robot service akan berada pada kondisi steady hingga perintah dari elderly diberikan. Deteksi orang dengan pose estimation digunakan untuk gerak dan gestur elderly. Pose estimation menggunakan arsitektur BlazePose memiliki keuntungan berupa model yang dihasilkan yang ringan. Sebuah model yang ringan sangat diunggulkan dalam perangkat komputasi rendah. Akan tetapi, model dari BlazePose memiliki kekurangan pada multi-person estimation. Selain itu, model ini akan mengestimasi pose single person yang terlihat pertama kali. Kondisi ini menjadi tidak menguntungkan elderly apabila terdapat beberapa orang di sekitarnya. Robot service akan mengestimasi pose orang yang salah.

% % Proposed method
% A model is required to solve this multi-person pose estimation problem. This model ensures that the person whose pose is being estimated does not switch to another person. Frame rate and inference time need to be considered for real-time pose estimation needs. Thus, this research proposes a design for predicting missing human pose using MediaPipe Pose Estimation in frames. Keypoint coordinates relative to the image will be stored to predict the position of individuals in a multi-person scenario. The multi-person detection in a frame will be performed using MediaPipe Object Detection, where each person in a frame will have their pose extracted and compared with the keypoint coordinates from the previous frame before they disappeared from the current frame.

% % Jadi, riset ini mengusulkan desain prediksi human pose estimation menggunakan MediaPie Pose Estimation yang hilang dari frame. Koordinat keypoint terhadap citra akan disimpan untuk memprediksi posisi orang pada multi-person. Deteksi multi-person pada sebuah frame akan dilakukan menggunakan MediaPipe Object Detection, dimana setiap orang pada sebuah frame akan diekstrak pose mereka kemudian dibandingkan dengan koordinat keypoint pada frame terakhir ketika sebelum orang ini menghilang dari frame.