\begin{spacing}{1}
    \input{./lib/identitasabstrakEng.tex}
    %Tulis Abstrak disini
    Older people's productivity often declines, especially in terms of physical ability. Physical decline can be slowed down with exercise and other physical activities. However, activities such as stretching are often neglected by the elderly. Exercise activities for the elderly are important so that the elderly can maintain their health in old age. Research on elderly activity recognition has been widely developed. Convolutional Neural Network (CNN) is a type of artificial neural network specifically designed for image processing and recognition. On the other hand, Long Time-Term Memory (LSTM) is an efficient method for solving real-time problems. These two methods can be used for labeling and recognizing physical activity movements in the elderly. Physical activities performed by the elderly need to be customized. In this study, we developed a model with elderly pose estimation. One of the frameworks for human pose estimation is Mediapipe Pose Estimation (MPE). Therefore, this research focuses on the recognition and detection of fitness movements in the elderly. The work begins with the acquisition of datasets in the form of exercise activities that have been adapted to the physical issues of the elderly. Data acquisition was carried out since there are few and limited datasets that discuss this training activity. The acquired video data was then subjected to a video frame extraction process. Each frame sequence represents the exercise activity information. Pose estimation has been done using the Mediapipe framework. The extraction results are then trained using CNN, LSTM, CNN-LSTM, and deep CNN-LSTM architectures. The accuracy of each model is 83.68\%, 92.89\%, 96.05\%, and 87.11\%. Based on these results, the CNN-LSTM model outperforms the other models with an accuracy rate of 96.05\%. The error in recognizing data patterns is shown using the loss metric. The loss value of the CNN-LSTM model is 0.1498, the smallest compared to other models. This value indicates the model's ability to predict data with the lowest error rate. In addition, in other metrics, this model outperforms other models. Precision, recall, and f1-score of the CNN-LSTM model are at 0.96, respectively.

    %Tulis Kata Kunci disini
    \vspace{2ex}
    \textbf{Keyword}: Activity, Deep Learning, Elderly, Pose Estimation
\end{spacing}