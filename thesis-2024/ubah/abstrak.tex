\begin{spacing}{1}
    \input{./lib/identitasabstrak.tex}
    %Tulis Abstrak disini
    Produktivitas orang tua sering kali menurun, terutama dalam hal kemampuan fisik. Penurunan kemampuan fisik dapat diperlambat dengan olahraga dan aktivitas fisik lainnya. Namun, aktivitas seperti peregangan sering diabaikan oleh orang tua. Aktifitas latihan untuk elderly ini menjadi penting agar elderly dapat menjaga kesehatannya di usia lanjut. Riset mengenai pengenalan aktivitas lansia telah banyak dikembangkan. Convolutional Neural Network (CNN) adalah jenis jaringan saraf tiruan yang dirancang khusus untuk pengolahan dan pengenalan gambar. Di sisi lain, Long Shoert-Term Memory (LSTM) adalah metode yang efisien untuk memecahkan masalah real-time. Kedua metode ini dapat digunakan untuk pelabelan dan pengenalan gerakan aktivitas fisik pada lansia. Aktivitas fisik yang dilakukan oleh lansia perlu disesuaikan. Dalam penelitian ini, kami mengembangkan sebuah model dengan estimasi pose lansia. Salah satu kerangka kerja untuk estimasi pose manusia adalah Mediapipe Pose Estimation (MPE). Oleh karena itu, penelitian ini berfokus pada pengenalan dan deteksi gerakan kebugaran pada lansia. Pekerjaan diawali dengan akuisisi dataset berupa aktifitas latihan yang telah disesuaikan dengan isu-isu fisik para elderly. Akusisi data dilakukan sejak sedikit dan terbatasnya dataset yang membahas aktifitas latihan ini. Data video yang telah diakuisisi kemudian dilakukan proses ekstraksi video frame. Setiap urutan frame mewakili informasi aktifitas latihan. Estimasi pose telah dilakukan menggunakan framework Mediapipe. Hasil ekstraksi ini kemudian dilatih menggunakan arsitektur CNN, LSTM, CNN-LSTM, dan deep CNN-LSTM. Akurasi setiap model sebesar 83.68\%, 92.89\%, 96.05\%, dan 87.11\%. Berdasarkan hasil tersebut, model CNN-LSTM mengungguli model-model lainnya dengan tingkat akurasi 96.05\%. Kesalahan dalam mengenali pola data ditunjukkan menggunakan metric loss. Nilai loss model CNN-LSTM sebesar 0.1498, paling kecil dibandingkan dengan model-model lainnya. Nilai ini menunjukkan kemampuan model dalam memprediksi data dengan tingkat kesalahan paling rendah. Selain itu, pada metrcis lainnya, model ini mengungguli daripada model lainnya. Precision, recall, dan f1-score model CNN-LSTM berada pada nilai 0.96, masing-masing.

    %Tulis Kata Kunci disini
    \vspace{2ex}
    \textbf{Kata kunci }: Aktivitas, \textit{Deep Learning}, Lansia, Estimasi Pose

\end{spacing}